\documentclass[notes]{beamer}          % print frame + notes
%\documentclass[notes=only]{beamer}     % only notes
%\documentclass{beamer}                 % only frames
 
\usecolortheme{beaver}

% Some commonly used packages
% (copied mainly from the Utrecht University theme: https://www.overleaf.com/project/5c900fa3bd9930036341116a)
\usepackage{ragged2e}  % `\justifying` text
\usepackage{booktabs}  % Tables
\usepackage{tabularx}
\usepackage{tikz}      % Diagrams
\usetikzlibrary{calc, shapes, backgrounds}
\usepackage{amsmath, amssymb, amsfonts, amsthm}
\usepackage{url}       % `\url`s
\usepackage{listings}  % Code listings
\usepackage{comment}  
\usepackage{mathtools}
\usepackage{graphicx}
\usepackage{subfig}
\usepackage{amsmath}

% Mainly math commands
\newcommand{\vect}[1]{\bm{#1}}
\usepackage{amsfonts}% to get the \mathbb alphabet
\newcommand{\field}[1]{\mathbb{#1}}
\newcommand{\C}{\field{C}}
\newcommand{\R}{\field{R}}
\newcommand{\norm}[1]{\left\lVert#1\right\rVert}
\newcommand{\argmin}{\operatornamewithlimits{argmin}}
\providecommand{\abs}[1]{\lvert#1\rvert}
\providecommand{\norm}[1]{\lVert#1\rVert}

% A variable used to exclude slides from the lecture version
\newif\iffull
%\fullfalse
\fulltrue

% Bibliography 
\usepackage[uniquename=init,giveninits=true,maxcitenames=1,style=authortitle-comp,backend=bibtex8]{biblatex}
%\bibliography{references}
\addbibresource{references.bib}

%Information to be included in the title page:
\title{Linear models}
\author{Federica Eduati}
\institute{Eindhoven University of Technology

Department of Biomedical Engineering}
\date{2021}
 
 
 
\begin{document}
 
\frame{\titlepage}
 
\begin{frame}
\frametitle{Learning goals}

At the end of this lecture you will:
\begin{itemize}
    \item Have a good understanding of linear models for regression, including methods for subset selection and shrinkage. 
    \item Have a good understanding of logistic regression as linear method for classification.
\end{itemize}

\vspace{5mm} 

Materials: 
\begin{itemize}
    \item Chapters 3 and 4 from \cite{elements}
\end{itemize}

\end{frame}


\begin{frame}{Overview}
Topics covered in this lecture:
    \tableofcontents
\end{frame}

\begin{frame}{Mentimeter}
Go to www.menti.com and use the code 67236648
\end{frame}



\section{Linear models for regression}


\begin{frame}
\frametitle{Introduction to linear models}
Linear models: a linear combination of the \textit{inputs} (also known as \textit{predictors}, \textit{features} or \textit{independent variables}) is used to predict one or more \textit{outputs} (also known as \textit{responses} or \textit{dependent variables}).

\vspace{5mm} 

The prediction task is defined as:
\begin{itemize}
    \item \textit{Regression}: when we predict \textit{quantitative outputs}.
    \item \textit{Classification}: when we predict \textit{qualitative outputs} (also referred to as \textit{categorical or discrete variables}).
\end{itemize}

\end{frame}

\begin{frame}
\frametitle{Mentimeter question (www.menti.com code 67236648)}

Excluding COVID-19 infection based on blood values is a:
\begin{itemize}
    \item regression problem
    \item classification problem
\end{itemize}
\end{frame}

\begin{frame}
\frametitle{Linear regression model}
Given a vector of inputs $X^T = (X_1, X_2, \dots, X_p)$, where $p$ is the number of features, a linear regression model has the form:

\begin{equation*}
f(X)=\beta_0 + \sum_{j=1}^p X_j\beta_j
\end{equation*}
where $\beta_j$ are the unknown parameters or coefficients. 

\vspace{5mm} 

Note that the model remains linear in the parameters even if the the variables $X_j$ are polynomial (e.g. $X_2 = X_1^2$) or derive from interactions between variables (e.g. $X_3 = X_1 \cdot X_2$).

\end{frame}

\begin{frame}
\frametitle{Estimation via least squares}
Parameters $\beta_j$ can be estimated from a set of training data $(x_1,y_1) \dots (x_N,y_N)$, where each $x_i=(x_{i1}, x_{i2}, \dots, x_{ip})^T$ is a vector of feature measurements for the $i$th case.

\vspace{5mm} 

With the \textit{least squares} estimation methods, the coefficients $\beta=(\beta_0, \beta_1, \dots, \beta_p)^T$ are selected by minimizing the residual sum of squares:

\begin{equation*}
RSS(\beta) = \sum_{i=1}^N (y_i - f(x_i))^2 = \sum_{i=1}^N (y_i - \beta_0 -  \sum_{j=1}^p x_{ij}\beta_j)^2
\end{equation*}
\end{frame}

\begin{frame}
\frametitle{Example of univariate linear regression}
$\hat{y} = \beta_0 + \beta_1 x$, where output variable $Y$ represents the sensitivity to a certain drug and input variable $X$ the expression of a gene.

\begin{center}
\includegraphics[height=6cm]{../figures/week_3/Univariate_linear_regression_0.pdf}
\end{center}

\end{frame}

\begin{frame}
\frametitle{Example of univariate linear regression}
Least square fitting: linear function of $X$ that minimises the sum of squared residuals.

\begin{center}
\includegraphics[height=6cm]{../figures/week_3/Univariate_linear_regression_1.pdf}
\end{center}

\end{frame}

\begin{frame}
\frametitle{Example of univariate linear regression}
Least square fitting: linear function of $X$ that minimises the sum of squared residuals. $RSS = \sum_{i=1}^N (y_i - \beta_0 - \beta_1 x_i)^2$

\begin{center}
\includegraphics[height=6cm]{../figures/week_3/Univariate_linear_regression_2.pdf}
\end{center}

\end{frame}

\begin{frame}
\frametitle{Genomics of Drug Sensitivity in Cancer (GDSC) project}
We will use as example data from the GDSC project (https://www.cancerrxgene.org/) \cite{GDSC}.

\begin{center}
\includegraphics[height=6cm]{../figures/week_3/GDSC_study_description.png}
\end{center}

\end{frame}

\begin{frame}
\frametitle{Genomics of Drug Sensitivity in Cancer (GDSC) project}
Cancer patients tends to respond very differently to drugs. A big challenge in cancer research 
is to find ways to assign the optimal treatment to each patient.

\begin{center}
\includegraphics[height=2.5cm]{../figures/week_3/precision_oncology.png}
\end{center}

The aim of the study is to find biomarkers of drug sensitivity.

\begin{center}
\includegraphics[height=2.5cm]{../figures/week_3/GDSC_study_aim.png}
\end{center}

\end{frame}

\begin{frame}
\frametitle{Our dataset}

We will consider data for 148 cell lines from four cancer types.

\begin{center}
\begin{tabular}{ |c|c| } 
 \hline
 ID & Cancer type \\
 \hline
 COAD/READ & Colorectal adenocarcinoma \\ 
 NB & Neuroblastoma \\ 
 KIRC & Kidney renal clear cell carcinoma \\ 
 BRCA & Breast carcinoma \\
 \hline
\end{tabular}
\end{center}

We can interpret the prediction of drug sensitivity as a linear regression problem using:

\begin{itemize}
    \item Output variable: Sensitivity to YM155 (Sepantronium bromide), as natural log of the fitted IC50. 
    \item Predictors: expression of 244 genes (the ones with higher variance), as RMA normalised expression.
\end{itemize}

\end{frame}

\begin{frame}
\frametitle{Feature scaling}

Idea: make sure features are on a similar scale.

\begin{equation*}
x_i = \frac{x_i-\mu_i}{\sigma_i}
\end{equation*}

where $\mu_i$ and $\sigma_i$ are respectively the average and the standard deviation of all the values of feature $i$.

\vspace{5mm}

This makes the estimated coefficients comparable and speeds up the optimization algorithm.

\vspace{5mm} 

If we standardize the predictors, the solution for $\hat{\beta_0}$ is $\overline{y}$, therefore we fit a model without the intercept.

\end{frame}


\section{Subset selection}

\begin{frame}
\frametitle{Model complexity}

Especially when we have a large number of features ($p$ large compared to $N$), least squares estimates can suffer from:

\begin{itemize}
    \item Low \textit{prediction accuracy}: high model complexity gives low bias but high variance, setting some coefficients to zero can reduce the variance of predictions (at the price of increasing the bias).
    \item Poor \textit{interpretability}: we might want to identify which predictors are really useful to explain the output.
\end{itemize}

\end{frame}


\begin{frame}
\frametitle{Mentimeter question (www.menti.com code 67236648)}

A model with a high number of parameters will tend to:
\begin{itemize}
    \item represent well the training set but not the test set
    \item represent well the test set but not the training set
    \item represent well both
\end{itemize}
\end{frame}


\begin{frame}
\frametitle{Model complexity and overfitting}

\begin{center}
\includegraphics[height=5.5cm]{../figures/week_3/Bias_variance_complexity.pdf}
\end{center}

Complex models tend to overfit the training data and perform poorly on the test data.

\end{frame}



\begin{frame}
\frametitle{Subset selection}

With \textit{subset selection} we want to retain only a subset of the predictors and eliminate the rest from the model.

\vspace{5mm} 

We can try a lot of different models, each containing a different subset of the predictors, and check which model is the best, e.g. based on \textit{Akaike information criterion (AIC)} or \textit{Bayesian information criterion (BIC)}.

\vspace{5mm} 

Problem: there are a total of $2^p$ models that contain subsets of p predictors!!

\end{frame}

\begin{frame}
\frametitle{Subset selection}

Three classical approaches to step-wise \textit{subset selection}:

\begin{itemize}
    \item \textit{Forward selection}. Start from the null model (i.e. only intercept) and sequentially add to the model the variable that gives lowest RSS (i.e. higest improvement of the fit).
    \item \textit{Backward selection}. Start with the full model (i.e. all variables) and sequentially remove the predictor with the largest p-value (i.e. lower impact on the fit).
    \item \textit{Mixed selection}. Alternation of forward and backward steps to add variables that improves RSS while maintaining the p-value below a certain threshold.
\end{itemize}

\end{frame}

\begin{frame}
\frametitle{Subset selection}

\textit{Subset selection} tends to:

\begin{itemize}
    \item improve \textit{interpretability}: only the most relevant predictors useful to explain the output are selected.
    \item still suffer from  low \textit{prediction accuracy}: high variance (discrete process in retaining and discarding predictors) often does not reduce prediction error.
\end{itemize}

\end{frame}


\section{Shrinkage methods}

\begin{frame}
\frametitle{Improving least square estimates with regularization}
\begin{itemize}
	\item least square estimates generally provide all non-zero coefficient.
	\item if $p>N$, solutions are not unique (i.e. multiple solutions with same minimum, often overfitting the data).
\end{itemize}


\vspace{5mm} 

Need to constrain or regularize the estimation process.

\vspace{5mm} 

Idea: shrink regression coefficients by imposing a penalty on their size.


\end{frame}

\begin{frame}
\frametitle{Ridge regression}
\textit{Ridge regression} penalizes the sum of squares of the coefficients (L2 regularization). 

\begin{align*}
\hat{\beta}^{ridge} &= \argmin_{\beta} \left\{ \sum_{i=1}^N (y_i - \beta_0 -  \sum_{j=1}^p x_{ij}\beta_j)^2 + \lambda \sum_{j=1}^p \beta_j^2 \right\} \\
& =  \argmin_{\beta} \left\{ RSS + \lambda \sum_{j=1}^p \beta_j^2 \right\}
\end{align*}

\end{frame}


\begin{frame}
\frametitle{Ridge regression: the meaning of $\lambda$}

\begin{equation*}
    \hat{\beta}^{ridge} =  \argmin_{\beta} \left\{ RSS + \lambda \sum_{j=1}^p \beta_j^2 \right\}
\end{equation*}

$\lambda \geq 0$ is a tuning parameter.

\begin{itemize}
    \item $\lambda = 0$ corresponds to the least square estimates.
    \item for larger values of $\lambda$, coefficients $\beta_1, \dots, \beta_p$ will be shrinked towards zero
\end{itemize}
\end{frame}

\begin{frame}
\frametitle{Ridge regression: the meaning of $\lambda$}


\begin{center}

\includegraphics[height=5.5cm]{../figures/week_3/Ridge_regression_coefficinets.pdf}
\end{center}

Profile of the Ridge regression coefficients for the GDSC example.

\end{frame}

\begin{frame}
\frametitle{The effect of regularization on bias and variance}
In matrix form we can rewrite the Ridge regression as:

\begin{equation*}
    (\mathbf{y} - \mathbf{X}\beta)^T (\mathbf{y} - \mathbf{X}\beta) + \lambda\beta^T\beta
\end{equation*}

And the Ridge regression solution becomes:

\begin{equation*}
    \hat{\beta}^{ridge} = (\mathbf{X}^T \mathbf{X} + \lambda \mathbf{I})^{-1}\mathbf{X}^T\mathbf{y} 
\end{equation*}

where $\mathbf{I}$ is the $p \times p$ identity matrix. This is linear in $\mathbf{y}$ and in closed form.

Assuming that $\mathbf{y} = \mathbf{X}\beta_{\mathrm{true}} + \mathbf{e}$, where $\mathbf{e}$ is a random error on the output variable, the Ridge regression solution can be written as:

\begin{align*}
    \hat{\beta}^{ridge} &= (\mathbf{X}^T \mathbf{X} + \lambda \mathbf{I})^{-1}\mathbf{X}^T(\mathbf{X}\beta_{\mathrm{true}} + \mathbf{e}) \\
    &= \underbrace{(\mathbf{X}^T \mathbf{X} + \lambda \mathbf{I})^{-1}\mathbf{X}^T\mathbf{X}\beta_{\mathrm{true}}}_{\textrm{bias}} + \underbrace{(\mathbf{X}^T \mathbf{X} + \lambda \mathbf{I})^{-1}\mathbf{X}^T\mathbf{e}}_{\textrm{variance}}
\end{align*}

\end{frame}

\begin{frame}
\frametitle{Mentimeter question (www.menti.com code 67236648)}

What is the value of $\lambda$ that gives the minimum variance?

\end{frame}


\begin{frame}
\frametitle{The effect of regularization on bias and variance}
Effect of $\lambda$ on the bias-variance trade off in a simulated example ($y_i = x_{i1}\beta_1 + x_{i2}\beta_2 + e_i$, $i=1,\dots, 20$). The model is simulated 50k times, each time the noise $e_i$ is randomly generated.




\begin{center}
\includegraphics[height=5.5cm]{../figures/week_3/Ridge_bias_variance_insilico.pdf}
\end{center}

\end{frame}


\begin{frame}
\frametitle{Lasso regression}
\textit{Lasso regression} penalizes the sum of the absolute value of the coefficients (L1 regularization). 

\begin{align*}
\hat{\beta}^{lasso} &= \argmin_{\beta} \left\{ \sum_{i=1}^N (y_i - \beta_0 -  \sum_{j=1}^p x_{ij}\beta_j)^2 + \lambda \sum_{j=1}^p | \beta_j | \right\} \\
& =  \argmin_{\beta} \left\{ RSS + \lambda \sum_{j=1}^p | \beta_j | \right\}
\end{align*}

\end{frame}


\begin{frame}
\frametitle{Lasso regression: the meaning of $\lambda$}

\begin{equation*}
    \hat{\beta}^{lasso} =  \argmin_{\beta} \left\{ RSS + \lambda \sum_{j=1}^p | \beta_j | \right\}
\end{equation*}

$\lambda \geq 0$ is a tuning parameter.

\begin{itemize}
    \item $\lambda = 0$ corresponds to the least square estimates
    \item larger values of $\lambda$, will make some of the coefficients $\beta_1, \dots, \beta_p$ to be exactly zero, performing a continuous subset selection.
\end{itemize}
\end{frame}

\begin{frame}
\frametitle{Mentimeter question (www.menti.com code 67236648)}

A high value of $\lambda$:
\begin{itemize}
    \item increases model complexity
    \item reduces model complexity
    \item doesn't affect model complexity
\end{itemize}
\end{frame}



\begin{frame}
\frametitle{Lasso regression: the meaning of $\lambda$}

\begin{center}
\includegraphics[height=5.5cm]{../figures/week_3/Lasso_regression_coefficinets.pdf}
\end{center}

Profile of the Lasso regression coefficients for the GDSC example.

\end{frame}

\begin{frame}
\frametitle{Comparing Lasso and Ridge regression}

An equivalent way or writing the Ridge and Lasso problems is:

\begin{equation*}
    \argmin_{\beta} \sum_{i=1}^N (y_i - \beta_0 -  \sum_{j=1}^p x_{ij}\beta_j)^2
\end{equation*}

subject to: 
\begin{itemize}
    \item $\sum_{j=1}^p \beta_j^2 < t$ for Ridge regression
    \item $\sum_{j=1}^p | \beta_j | < t$ for Lasso regression
\end{itemize}

\vspace{5mm} 

It is possible to show that there is a one-to-one correspondence between the parameter $\lambda$ and $t$.

\end{frame}


\begin{frame}
\frametitle{Comparing Lasso and Ridge regression}

\begin{center}
\includegraphics[height=7cm]{../figures/week_3/Comparison_lasso_ridge.pdf}
\end{center}

\end{frame}


\begin{frame}
\frametitle{How to select optimal $\lambda$}

We need a method to select the tuning parameter $\lambda$ for both Lasso and Ridge regression.

\vspace{5mm} 

We want the model that provides the best predictions on the test set. In general this is very sensitive to the data used for training.
\end{frame}


\begin{frame}
\frametitle{How to select optimal $\lambda$}

Two data partitions - GDSC example (90\% training, 10\% test).

\begin{center}
\includegraphics[height=7cm]{../figures/week_3/Lasso_MSE_train_test.pdf}
\end{center}
\end{frame}

\begin{frame}
\frametitle{How to select optimal $\lambda$}

We can use cross-validation following these steps:
\begin{enumerate}
  \item Choose a grid of $\lambda$.
  \item Optimize the model for each training set and compute Mean-Squared Error (MSE) for each validation set.
  \item Choose the model with smallest cross-validation error.
  \item Re-fit using all the available observations and the selected tuning parameter.
\end{enumerate}

\end{frame}

\begin{frame}
\frametitle{$\lambda$ selection: GDSC example 10-fold cross validation}

\begin{center}
\includegraphics[height=5cm]{../figures/week_3/Lasso_MSE_crossvalidation.pdf}
\end{center}

\vspace{-2mm} 

\begin{itemize}
    \item $\lambda_{min}$: value of $\lambda$ that gives the minimum MSE
    \item $\lambda_{min+1se}$: largest value of $\lambda$ such that error is within 1 standard error of the minimum
\end{itemize}

\end{frame}

\begin{frame}
\frametitle{Mentimeter question (www.menti.com code 67236648)}

Would you select $\lambda_{min}$ or $\lambda_{min+1se}$ to have a less complex model'?

\end{frame}

\begin{frame}
\frametitle{Lasso for feature selection}
GDSC example 10-fold cross validation

\begin{center}
\includegraphics[height=5cm]{../figures/week_3/Lasso_feature_selection.pdf}
\end{center}

$\lambda_{min+1se}$ gives more sparse solutions.

\end{frame}


\begin{frame}
\frametitle{Lasso for feature selection}

Selected features can be sensitive to the partition of the cross-validation.

\vspace{5mm} 

To select only robust features we can use \textit{bootstrap}: feature selection procedure is repeated M times, using every time a different bootstrapped dataset (i.e. sampling N samples with replacement).

\end{frame}


\begin{frame}
\frametitle{Lasso for feature selection: using bootstrap}
GDSC example: 100 times bootstrap (with 10-fold cross-validation)

\begin{center}
\includegraphics[height=7cm]{../figures/week_3/Lasso_feature_selection_bootstrap.pdf}
\end{center}

\end{frame}




\begin{frame}
\frametitle{Collinearity}
Two or more predictor variables can be closely related to each other and show a high correlation.

\vspace{5mm} 

This is common in genomic data, because genes can be redundant or have related functions.

\begin{center}
\includegraphics[height=5cm]{../figures/week_3/collinearity.pdf}
\end{center}

\end{frame}

\begin{frame}
\frametitle{Collinearity}
The effect of \textit{collinear variables} can be difficult to separate in the regression context.

\vspace{5mm} 

\textit{Collinearity} in general reduces the accuracy of the estimates of the coefficients of the correlated variables.

\vspace{5mm} 

When using Lasso, \textit{collinearity} can result in an arbitrary selection of one or the other correlated feature, depending on the data used for training.
\end{frame}

\begin{frame}
\frametitle{Mentimeter question (www.menti.com code 67236648)}

If you do bootstrap with lasso with 2 explanatory features which are collinear, what do you expect as frequency of selection?

\end{frame}

\begin{frame}
\frametitle{Elastic Net regression}

Elastic Net provides a compromise between Ridge and Lasso, by selecting variables like Lasso and shrinking together correlated variables like Elastic Net.


\begin{equation*}
    \hat{\beta}^{elastic net} =  \argmin_{\beta} \left\{ RSS + \lambda \sum_{j=1}^p ((\alpha \beta_j^2 + (1-\alpha)| \beta_j |) \right\}
\end{equation*}

Where $\alpha$ is a tunable parameter. This corresponds to:

\begin{itemize}
    \item Ridge for $\alpha=1$
    \item Lasso for $\alpha=0$
\end{itemize}

\end{frame}

\begin{frame}
\frametitle{Elastic Net regression}

\begin{center}
\includegraphics[height=5.5cm]{../figures/week_3/ElasticNet_regression_coefficinets.pdf}
\end{center}

Profile of the Elastic net (with $\alpha=0.5$) regression coefficients for the GDSC example.

\end{frame}

\section{Classification}

\begin{frame}
\frametitle{Classification}
In a clinical setting, instead of predicting the continuous IC50 for each patient, we might be interested in classifying patients in sensitive or resistant to a specific drug.

\vspace{5mm} 

Using the cell lines from the GDSC study this corresponds to divide them in sensitive and resistant to the drug. 

\begin{center}
\includegraphics[height=2.7cm]{../figures/week_3/GDSC_classification_overview.pdf}
\end{center}

Cell lines are separated in sensitive and resistant based on the average response across all cell lines in the panel.

\end{frame}

\begin{frame}
\frametitle{Classification}

For binary classification we will consider a positive and a negative class:

\begin{itemize}
    \item 0 ("negative class"): sensitive
    \item 1 ("positive class"): resistant
\end{itemize}

\begin{center}
\includegraphics[height=5cm]{../figures/week_3/GDSC_classification_sensitivity.pdf}
\end{center}
\end{frame}


\begin{frame}
\frametitle{Example of classification with one variable}

Using the same example used for univariate linear regression, we consider the ability of gene ABCB1 to classify cell lines in sensitive and resistant.

\begin{center}
\includegraphics[height=4cm]{../figures/week_3/GDSC_one_variable_example_logistic_regression_1.pdf}
\end{center}

\end{frame}

\begin{frame}
\frametitle{Example of classification with one variable}

We can try to fit a linear regression to a binary response, e.g.using 0.5 as threshold for prediction.

\begin{center}
\includegraphics[height=4.5cm]{../figures/week_3/GDSC_one_variable_example_logistic_regression_2.pdf}
\end{center}

Poor interpretability, with predicted values also outside [0, 1].

\end{frame}

\begin{frame}
\frametitle{Example of classification with one variable}

With logistic regression we want to model the probability (between 0 and 1) that $Y$ belongs to a particular class.

\begin{equation*}
Pr(response = resistant|ABCB1expression)
\end{equation*}


\begin{center}
\includegraphics[height=4.5cm]{../figures/week_3/GDSC_one_variable_example_logistic_regression_3.pdf}
\end{center}

\end{frame}


\section{Logistic regression}
\begin{frame}
\frametitle{Logistic regression}

\begin{equation*}
p(X) = Pr(G = 1|X)
\end{equation*}

With \textit{logistic regression} this probability is modeled as a \textit{logistic function}:

\begin{equation*}
p(X) = \frac{e^{\beta_0 + \beta_1X}}{1+e^{\beta_0 + \beta_1X}}
\end{equation*}

This corresponds to:

\begin{equation*}
\log \left(\frac{p(X)}{1-p(X)}\right)  = \beta_0 + \beta_1X
\end{equation*}

The left-hand side is called the log-odds or logit transformation and it is linear in X.

\end{frame}

\begin{frame}
\frametitle{Logistic regression: multi-class classification}

When having $K$ classes, the posterior probability is modeled as:

\begin{align*}
\log \frac{Pr(G=1 | X=x)}{Pr(G=K | X=x)} &= \beta_{10} + \beta_1^T x \\
\log \frac{Pr(G=2 | X=x)}{Pr(G=K | X=x)} &= \beta_{20} + \beta_2^T x \\
 & \vdots \\
\log \frac{Pr(G=K-1 | X=x)}{Pr(G=K | X=x)} &= \beta_{(K-1)0} + \beta_{K-1}^T x \\
\end{align*}

Using $K$-1 log-odds satisfies the constraint that probabilities sum to one. The choice of the last class to be used at denominator is arbitrary and does not affect the result.

\end{frame}



\begin{frame}
\frametitle{Logistic regression: parameters estimation}

The parameters for logistic regression models are generally estimated using \textit{maximum-likelihood}.

\vspace{5mm} 

\vspace{5mm} 

For $N$ observations the log-likelihood to be maximized is:

\begin{equation*}
    \ell(\theta) = \sum_{i=1}^N \log p_{g_i}(x_i; \theta)
\end{equation*}

where $p_k(x_i; \theta) = Pr(G=k|X=x_i; \theta)$ and $\theta$ is the parameter set.


\end{frame}

\begin{frame}
\frametitle{Logistic regression: parameters estimation}

For the two classes case, the response $y_i \in \{0,1\}$. Being $p(x_i;\beta)$ the probability for class 1, and $1-p(x_i;\beta)$ the probability for class 0, the log-likelihood can be written as:

\begin{align*}
    \ell(\beta) &= \sum_{i=1}^N \{ y_i \log p(x_i;\beta) + (1-y_i) \log (1-p(x_i;\beta)) \} \\
    &= \sum_{i=1}^N \{ y_i \beta^T x_i - \log (1+e^{\beta^T x_i}) \}
\end{align*}

where $\beta=\{ \beta_{10}, \beta_1\}$ and the vector of inputs $x_i$ includes the constant term 1 for the intercept.


\end{frame}



\begin{frame}
\frametitle{L1 regularized logistic regression}

Similarly to what we have seen for linear regression, regularization can be used for variable selection and shrinkage also with logistic regression.

\vspace{5mm} 

When applying L1 penalty we want to maximise a penalized version of the log-likelihood

\begin{equation*}
    \max_{\beta_0, \beta} \left\{ \sum_{i=1}^N [ y_i(\beta_0 + \beta^T x_i) - \log (1+e^{(\beta_0 + \beta^T x_i))}] -\lambda \sum_{j=1}^p | \beta_j| \right\}
\end{equation*}

Note that the intercept term is not penalized. Similarly to Lasso, the predictors should be standardized in order for the coefficients to be comparable.

\end{frame}

\begin{frame}
\frametitle{Example cross-validation}

Also in this case cross-validation can be used to select the tuning parameter.

\begin{center}
\includegraphics[height=4.5cm]{../figures/week_3/GDSC_logistic_regression_cross_validation.pdf}
\end{center}

\end{frame}


\begin{frame}
\frametitle{References}
\printbibliography
\end{frame}


\end{document}
